\pdfoutput =0\relax 
%%%%%%%%%%%%%%%%%%%%%%%%%%%%%%%%%%%%%%%%%
% Minimal Invoice
% LaTeX Template
% Version 1.1 (April 22, 2022)
%
% This template originates from:
% https://www.LaTeXTemplates.com
%
% Author:
% Vel (vel@latextemplates.com)
%
% License:
% CC BY-NC-SA 4.0 (https://creativecommons.org/licenses/by-nc-sa/4.0/)
%
%%%%%%%%%%%%%%%%%%%%%%%%%%%%%%%%%%%%%%%%%

%----------------------------------------------------------------------------------------
%	CLASS, PACKAGES AND OTHER DOCUMENT CONFIGURATIONS
%----------------------------------------------------------------------------------------

\documentclass[
	a4paper, % Paper size, use 'a4paper' for A4 or 'letterpaper' for US letter
	9pt, % Default font size, available sizes are: 8pt, 9pt, 10pt, 11pt, 12pt, 14pt, 17pt and 20pt
]{CSMinimalInvoice}
\taxrate{12.5}

% \usepackage{auto-pst-pdf}
\usepackage{pstool}
\usepackage{pstricks}
\usepackage{qrcode}
\usepackage{pst-barcode}

% The currency code (e.g. USD is United States Dollars), do one of the following:
% 1) Enter a 3 letter code to have it appear at the bottom of the invoice
% 2) Leave the command empty (i.e. \currencycode{}) if you don't want the code to appear on the invoice
\currencycode{USD}

% The default currency symbol for the invoice is the dollar sign, if you would like to change this, do one of the following:
% 1) Uncomment the line below and enter one of the following currency codes to change it to the corresponding symbol for that currency: GBP, CNY, JPY, EUR, BRL or INR
%\determinecurrencysymbol{GBP}
% 2) Uncomment the line below and leave it blank for no currency symbol or use another character/symbol for your currency
%\renewcommand{\currencysymbol}{}

% The invoice number, do one of the following:
% 1) Enter an invoice number, it may include any text you'd like such as '13-A'
% 2) Leave command empty (i.e. \invoicenumber{}) and no invoice number will be output in the invoice
\invoicenumber{40}

%---------------------------------------------------------------------------------
%	ADVANCED INVOICE SETTINGS
%---------------------------------------------------------------------------------

\roundcurrencytodecimals{2} % The number of decimal places to round currency numbers
\roundquantitytodecimals{2} % The number of decimal places to round quantity numbers

% Advanced settings for changing how numbers are output
\sisetup{group-minimum-digits=4} % Delimit numbers (e.g. 4000 -> 4,000) when there are this number of digits or more
\sisetup{group-separator={,}} % Character to use for delimiting digit groups
\sisetup{output-decimal-marker={.}} % Character to use for specifying decimals

\currencysuffix{} % Some currencies output the currency symbol after the number, such as Sweden's krona specified with a 'kr' suffix. Specify a suffix here if required, otherwise leave this command empty.

%---------------------------------------------------------------------------------

\usepackage [active,tightpage]{preview}
\pagestyle {empty}

\makeatletter 
\@input {template.oldaux}
\makeatother 

\begin {document}
\makeatletter 
\immediate \write \@mainaux {\@percentchar <*PSTOOLLABELS>}
\makeatother 

\centering \null \vfill 

\begin {preview}
 \psset {unit=1bp} \begin {pspicture}(252,86) \psbarcode {12345678}{includetext inkspread=0.5}{ean8} \end {pspicture} 
 \includegraphics [width=3.5in] {barcode}
\end {preview}

\vfill 

\makeatletter 
\immediate \write \@mainaux {\@percentchar </PSTOOLLABELS>}
\makeatother 

\end {document}

